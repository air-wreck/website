\documentclass{article}

\usepackage{amsmath}
\usepackage{amsfonts}
\usepackage{amssymb}
\usepackage{amsthm}
\usepackage{mathtools}
\usepackage{tcolorbox}

% set up color box theorems
\tcbuselibrary{theorems}
\newtcbtheorem[number within=section]{theorem}{Theorem}
  {
    colback=blue!5,
    colframe=blue!35!black,
    fonttitle=\bfseries
  }{thm}

% set up pretty remarks
\theoremstyle{definition}
\newtheorem*{remark}{Remark}

% math definitions: I expect more will be added as I go along
\def\gcd{\textrm{gcd}}
\def\divides{\mid}

\title{15-151 Midterm 2 Review}
\author{Eric Zheng}
\date{October 22, 2019}

\begin{document}
\maketitle

\section{Named Theorems and Important Results}
\begin{theorem}{Bezout's lemma}{bezout}
  For all $a,b,c \in \mathbb{Z}$, the equation $ax+by=c$ has a solution
  $x,y \in \mathbb{Z}$ if and only if $\gcd(a,b) \divides c$.
\end{theorem}
\begin{proof}
  We will prove the bi-implication in each direction separately.
  \begin{itemize}
    \item ($\Longrightarrow$) Suppose $\exists x,y \in \mathbb{Z}$ such that
      $ax+by=c$. By definition, $\gcd(a,b)$ must divide both $a$ and $b$, so
      we can write:
    \begin{equation*}
      n\cdot\gcd(a,b) = a \quad\quad m\cdot\gcd(a,b) = b
    \end{equation*}
    for some $n,m \in \mathbb{Z}$. So we have:
    \begin{align*}
      c &= ax+by                               \\
        &= nx\cdot\gcd(a,b) + my\cdot\gcd(a,b) \\
        &= (nx+my)\cdot\gcd(a,b)
    \end{align*}
    Then $\gcd(a,b) \divides c$, as required.
  \item ($\Longleftarrow$) We will prove this for $a,b \in \mathbb{N}$. Since
    $x,y \in \mathbb{Z}$, the result holds in general. Consider the following
    predicate:
    \begin{equation*}
      p(n) \coloneqq \textrm{``if $a+b=n$ and $\gcd(a,b) \divides c$, then
        $\exists x,y \in \mathbb{Z}$ such that $ax+by=c$''}
    \end{equation*}
    We proceed by induction on $n=a+b$.
    \begin{itemize}
      \item \textit{Base case.} If $n=0$, then $a=b=0$, so $\gcd(a,b)=0$. Now
        if $0 \divides c$, then $c=0$. Any $x,y \in \mathbb{Z}$ will satisfy
        $ax+by=c$.
      \item \textit{Induction step.} Let $n \ge 0$ be given, and assume that
        $p(k)$ holds for all $k \le n$. Let $a,b \in \mathbb{N}$ such that
        $a+b = n+1$, and suppose $c \in \mathbb{Z}$ such that
        $\gcd(a,b) \divides c$. Then there are three cases:
        \begin{enumerate}
          \item Case $a=b=0$. This case is not possible, since $n+1 > 0$.
          \item Case one of $a$ or $b$ is zero. Without loss of generality, let
            $a=0$. Then $\gcd(a,b)=b$, so by the induction hypothesis,
            $b \divides c$. By definition, $\exists y \in \mathbb{Z}$ such that
            $by = c$.
          \item Case $a,b > 0$. Without loss of generality, let $b \ge a$. Note
            that $a + (b-a) = b$, and $b \in [n]$. Additionally, since
            $\gcd(a,b-a)=\gcd(a,b)$, we have $\gcd(a,b-a) \divides c$. Now since
            $p(b)$ is true, there must exist some $x_0,y_0 \in \mathbb{Z}$ such
            that $ax_0+(b-a)y_0 = c$. But this implies that
            $a(x_0-y_0) + by_0 = c$, so we have found a solution to $ax+by=c$.
        \end{enumerate}
    \end{itemize}
  \end{itemize}
\end{proof}

\begin{theorem}{coprime (Euclid's) lemma}{coprime}
  For all $a,b,c \in \mathbb{Z}$, if $\gcd(a,b)=1$, then $a \divides bc$
  implies that $a \divides c$.
\end{theorem}
\begin{proof}
  Let $a,b,c \in \mathbb{Z}$ such that $\gcd(a,b)=1$. Now suppose that
  $a \divides bc$. By Bezout's lemma (Theorem \ref{thm:bezout}),
  $\exists x,y \in \mathbb{Z}$ such that $ax+by=1$. This implies that
  $cax+cby=c$. Clearly, $a \divides cax$, and by our assumption,
  $a \divides cby$. Thus, $a$ divides their sum, or $a \divides c$, as required.
\end{proof}

\begin{theorem}{solutions to linear Diophantine equations}{lindioph}
  For all $a,b,c \in \mathbb{Z}$, if $x_0, y_0 \in \mathbb{Z}$ satisfy
  $ax_0 + by_0 = c$, then for some $x,y \in \mathbb{Z}$, $ax+by=c$ if and only
  if $x$ and $y$ are of the form:
  \begin{equation*}
    x = x_0 + \frac{b}{\gcd(a,b)} \cdot k\quad \quad
    y = y_0 - \frac{a}{\gcd(a,b)} \cdot k
  \end{equation*}
  for some $k \in \mathbb{Z}$. (We have implicitly assumed that $\gcd(a,b) \neq
  0$, which is true if at least one of $a$ and $b$ is nonzero.)
\end{theorem}
\begin{proof}
  We will prove each direction of the bi-implication separately.
  \begin{itemize}
    \item ($\Longrightarrow$) Suppose $x_0,y_0 \in \mathbb{Z}$ satisfy
      $ax_0+by_0=c$. Now consider another pair $x,y \in \mathbb{Z}$ which also
      satisfy $ax+by=c$. Now note:
      \begin{align*}
                 & ax+by=c=ax_0+by_0                                       \\
        \implies & a(x-x_0) = b(y_0-y)                                     \\
        \implies & \frac{a}{\gcd(a,b)}(x-x_0) = \frac{b}{\gcd(a,b)}(y_0-y)
      \end{align*}
      Thus, we have:
      \begin{align*}
        \frac{a}{\gcd(a,b)} & \divides \frac{b}{\gcd(a,b)}(y_0-y) \\
        \frac{b}{\gcd(a,b)} & \divides \frac{a}{\gcd(a,b)}(x-x_0)
      \end{align*}
      But observe that:
      \begin{equation*}
        \gcd\left(\frac{a}{\gcd(a,b)}, \frac{b}{\gcd(a,b)}\right) = 1
      \end{equation*}
      So by Euclid's lemma (Theorem \ref{thm:coprime}), we must have:
      \begin{align*}
        \frac{a}{\gcd(a,b)} \divides (y_0 - y) \implies
          & y = y_0 - \frac{a}{\gcd(a,b)} \cdot k        \\
        \frac{b}{\gcd(a,b)} \divides (x - x_0) \implies
          & x = x_0 + \frac{b}{\gcd(a,b)} \cdot k
      \end{align*}
      for some $k \in \mathbb{Z}$. (It technically remains to be shown that the
      $k$ in the expressions for $x$ and $y$ are the same, but this is easy to
      do by contradiction.)
    \item ($\Longleftarrow$) Let $x_0, y_0 \in \mathbb{Z}$ satisfy
      $ax_0 + by_0 = c$, and consider some arbitrary $x,y$ of the form:
      \begin{align*}
        x &= x_0 + \frac{b}{\gcd(a,b)} \cdot k \\
        y &= y_0 - \frac{a}{\gcd(a,b)} \cdot k
      \end{align*}
      for some $k \in \mathbb{Z}$. Now consider the linear combination $ax+by$:
      \begin{align*}
        ax + by &= a\left(x_0 + \frac{b}{\gcd(a,b)} \cdot k\right)
                 + b\left(y_0 - \frac{a}{\gcd(a,b)} \cdot k\right)           \\
                &= ax_0 + by_0 + \frac{abk}{\gcd(a,b)}-\frac{abk}{\gcd(a,b)} \\
                &= ax_0 + by_0                                               \\
                &= c
      \end{align*}
  \end{itemize}
\end{proof}

\begin{theorem}{Wilson's theorem}{wilson}
  If $p \in \mathbb{N}$ is prime, then $(p-1)! \equiv -1 \bmod p$.
\end{theorem}
\begin{proof}
  Observe that:
  \begin{equation*}
    (p-1)! = (p-1)(p-2)(p-3)\dots(3)(2)(1)
  \end{equation*}
  Since $(p-1)(1) \equiv -1 \bmod p$, it suffices to show instead that
  \begin{equation*}
    (p-2)(p-3)\dots(3)(2) \equiv 1 \bmod p
  \end{equation*}
  First, we note that since each term $a_k = (p-k)$ in this product is coprime
  with $p$, by Bezout's lemma (Theorem \ref{thm:bezout}) it must have some
  multiplicative inverse among the factors. That is, there must be an integer
  solution to $a_kx + py = 1$ for all $2 \le k \le p - 2$. By definition, we
  have $a_kx \equiv 1 \bmod p$, so $x$ is the multiplicative inverse of the
  term $a_k$. Under modular arithmetic, we can take $0 \le x < p-1$, yet $x$
  cannot be $0$, $1$, or $p-1$. Thus, every term $a_k$ has a multiplicative
  inverse that is another term $a_i$.

  Next, we show that no term $a_k$ is its own inverse. If we had
  $a_k^2 \equiv 1 \bmod p$, then $a_k^2-1 \equiv (a_k+1)(a_k-1) \equiv 0
  \bmod p$. Since $p$ is prime, this implies that $a_k \equiv \pm 1 \bmod p$.
  (This was a homework exercise!) But $a_k \not\equiv \pm 1 \bmod p$ if we take
  $k$ between $2$ and $p-2$, so the inverse of each $a_k$ must be distinct from
  $a_k$ itself.

  From these two results, it follows that we can pair each term $a_k$ with its
  multiplicative inverse, so the resulting product must be $1$, as required.
\end{proof}

\begin{theorem}{Fermat's little theorem}{fermat}
  If $p \in \mathbb{N}$ is prime, then for all $a \in \mathbb{Z}$, $a^p \equiv
  a \bmod p$. If we additionally have $\gcd(a,p) = 1$, then $a^{p-1} \equiv 1
  \bmod p$.
\end{theorem}
\begin{proof}
  We will prove the second version of this theorem (which requires that
  $\gcd(a,p) = 1$). Let $p \in \mathbb{N}$ be prime, and denote $S = \{a, 2a,
  \dots, (p-1)a\}$. We note two things:
  \begin{enumerate}
    \item No distinct $x,y \in S$ satisfy $x \equiv y \bmod p$. To show this,
      assume for the sake of contradiction that $ma \equiv na \bmod p$ for some
      $m,n \in [p-1]$, yet $m \neq n$. Then $p \divides a(m-n)$, so we must have
      $p \divides a$ or $p \divides m-n$ since $p$ is prime. But $\gcd(a,p)=1$,
      so $p \nmid a$, and $m-n \in [p-1]$, so $p \nmid m-n$ by irreducibility.
      This is a contradiction, so our assumption is false.
    \item For each $x \in S$, $\exists y \in [p-1]$ such that $x \equiv y
      \bmod p$. We note that, under modulus $p$, each $x \in S$ must be
      congruent to some $0 \le y < p$. But observe that $y \neq 0$, since
      otherwise, it would follow that $p \divides x$. Again, this is not
      possible by the irreducibility of $p$ (see Theorem \ref{thm:prime-irr}).
  \end{enumerate}
  Now consider the product $a^{p-1}(p-1)!$ of all the elements in $S$. As we
  have shown, each distinct $x \in S$ is congruent to some distinct $y \in
  [p-1]$, so the elements are congruent to a permutation of $[p-1]$. It follows
  that:
  \begin{equation*}
    a^{p-1}(p-1)! \equiv (p-1)! \bmod p
  \end{equation*}
  And since, by Bezout's lemma (Theorem \ref{thm:bezout}), each $y \in [p-1]$
  has some multiplicative inverse, this is equivalent to saying:
  \begin{equation*}
    a^{p-1} \equiv 1 \bmod p
  \end{equation*}
  as required.
\end{proof}

\begin{theorem}{Euler's theorem}{euler}
  For all $a \in \mathbb{Z}$, $n \in \mathbb{N}$, if $\gcd(a,n) = 1$, then
  $a^{\varphi(n)} \equiv 1 \bmod n$.
\end{theorem}

\begin{remark}
  Note that Fermat's little theorem (Theorem \ref{thm:fermat}) is a special case
  of Euler's theorem (Theorem \ref{thm:euler}). Euler's theorem will not be on
  the exam, but it can't hurt to know it.
\end{remark}

\section{Other Interesting Results from Class}
\begin{theorem}{equivalence of primality and irreducibility}{prime-irr}
  An integer $p$ is prime if and only if it is irreducible.
\end{theorem}
\begin{proof}
  We will prove each direction of the bi-implication separately.
  \begin{itemize}
    \item ($\Longrightarrow$) Let $p \in \mathbb{Z}$ be a prime number. Now
      suppose we can express $p$ as $p = ab$ for some $a,b \in \mathbb{Z}$. Then
      $p \divides ab$, so by definition, $p \divides a$ or $p \divides b$.
      Without loss of generality, let $p \divides a$. Now $pq = a$ for some
      $q \in \mathbb{Z}$, so $p = ab \implies p = pqb \implies 1 = qb$. It
      follows that $b$ must be a unit, so $p$ is irreducible.
    \item ($\Longleftarrow$) Let $p \in \mathbb{Z}$ be irreducible, and suppose
      that $p \divides ab$. Now there are two cases:
      \begin{itemize}
        \item Case $p \divides a$. In this case, $p$ is prime by definition.
        \item Case $p \nmid a$. In this case, note that, by irreducibility, the
          only factors of $p$ are $\{\pm 1, \pm p\}$. Since $p \neq a$
          (otherwise, $p \divides a$), we must have $\gcd(p,a)=1$. Then by
          Euclid's lemma (Theorem \ref{thm:coprime}), we must have
          $p \divides b$, so $p$ is prime by definition.
      \end{itemize}
  \end{itemize}
\end{proof}

\begin{theorem}{existence of prime factorization}{exist-fact}
  For all $n \in \mathbb{N}$ such that $n \ge 2$, $n$ can be factored into the
  product of prime numbers.
\end{theorem}
\begin{proof}
  Let $p(n) \coloneqq \textrm{``$n$ can be factored into the product of prime
  numbers''}$. We will proceed by strong induction on $n$.
  \begin{itemize}
    \item \textit{Base case.} Consider $n=2$. Since $2$ is prime, $p(2)$ holds.
    \item \textit{Induction step.} Let $n \ge 2$ be given, and assume that
      $p(k)$ holds for all $2 \le k \le n$. Now consider $n+1$. There are two
      possibilities:
      \begin{enumerate}
        \item Case $n+1$ is prime. Then $p(n+1)$ is true.
        \item Case $n+1$ is not prime. Then we can write $n+1 = ab$ for some
          non-unit, non-zero $a, b \in \mathbb{N}$. But since $2 \le a,b \le n$,
          we know that $a$ and $b$ factor into primes by invoking the induction
          hypothesis. Thus, $n+1 = ab$ must factor into primes.
      \end{enumerate}
  \end{itemize}
\end{proof}

\begin{theorem}{uniqueness of prime factorization}{uniq-fact}
  For all $n \in \mathbb{N}$ such that $n \ge 2$, $n$ can be uniquely factored
  into the product of prime numbers.
\end{theorem}
\begin{proof}
  Consider the predicate:
  \begin{equation*}
    p(n) \coloneqq \textrm{``$n$ can be factored uniquely into the product of
                              primes''}
  \end{equation*}
  We proceed by strong induction on $n \ge 2$.
  \begin{itemize}
    \item \textit{Base case.} Let $n=2$. There is only one way to factor $2$ 
      into primes, namely $p_1=q_1=2$. Thus, $p(2)$ holds.
    \item \textit{Induction step.} Let $n \ge 2$ be given, and assume that
      $p(i)$ holds for all $2 \le i \le n$. Consider two prime factorizations
      written in non-decreasing order:
      \begin{align*}
        n+1 &= p_1 \cdot p_2 \cdot p_3 \cdots p_k \\
            &= q_1 \cdot q_2 \cdot q_3 \cdots q_l
      \end{align*}
      Without loss of generality, let $p_1 \le q_1$. Now since $p_1$ is prime
      and divides $q_1 \cdot q_2 \cdots q_l$, we must have $p_1 = q_i$ for
      some $i \in [l]$. But since the $q$'s are in non-decreasing order, we must
      have $q_1 \le q_i = p_1$. Since $p_1 \le q_1 \le p_1$, we must have $p_1 =
      q_1$. Then:
      \begin{equation*}
        p_2 \cdot p_3 \cdots p_k = q_2 \cdot q_3 \cdots q_l
      \end{equation*}
      Now there are two possibilities:
      \begin{enumerate}
        \item If there are no more prime factors, then we have shown $p(n+1)$ to
          be true.
        \item Otherwise, $2 \le p_2 \cdot p_3 \cdots p_k \le n$. In this case,
          we invoke the induction hypothesis to show that $p(n+1)$ is true.
      \end{enumerate}
  \end{itemize}
\end{proof}

\begin{theorem}{divisibility tricks}{div-tricks}
  Let $n \in \mathbb{N}$ have the base-ten expansion $d_0d_1d_2 \dots d_r$. We
  claim:
  \begin{enumerate}
    \item $n \equiv \sum _{i=0} ^{r} d_i \bmod 3$
    \item $n \equiv \sum _{i=0} ^{r} d_i \bmod 9$
    \item $n \equiv \sum _{i=0} ^{r} (-1)^{i} d_i \bmod 11$
  \end{enumerate}
\end{theorem}
\begin{proof}
  Note that the decimal expansion satisfies:
  \begin{equation*}
    n = \sum _{i=0} ^r d_i 10^i
  \end{equation*}
  Now examining each trick:
  \begin{enumerate}
    \item Observe that $10 \equiv 1 \bmod 3$, so $10^k \equiv 1^k \equiv 1
      \bmod 3$ for all $k \in \mathbb{N}$.
    \item Observe that $10 \equiv 1 \bmod 9$, so $10^k \equiv 1^k \equiv 1
      \bmod 9$ for all $k \in \mathbb{N}$.
    \item Observe that $10 \equiv -1 \bmod 11$, so $10^k \equiv (-1)^k \bmod 11$
      for all $k \in \mathbb{N}$.
  \end{enumerate}
\end{proof}

\section{Eric's Personal Reminders}
Here are a few things that I tend to forget easily:
\begin{enumerate}
  \item Don't gloss over induction mechanics! In particular, remember to define
    the predicate as $p(n) \coloneqq \textrm{``$\dots$''}$. Also quantify stuff
    where appropriate.
  \item Don't forget the exponent in Fermat's little theorem (Theorem
    \ref{thm:fermat}) is $p-1$, \textbf{not} $p$. On a similar note, the theorem
    has the requirement that $a$ and $p$ be coprime.
  \item Go over the divisibility tricks before the exam!
\end{enumerate}
\end{document}
